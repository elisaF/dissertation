\chapter{Introduction}
\label{ch:intro}


Understanding language often requires analysis beyond the word and sentence. Computational discourse fills this gap by describing how a text is structurally organized and how units of the text (such as sentences and clauses) are related to each other. We as humans make easy and ample use of discourse-level information: when you skim to the end of this chapter to find the meat of this thesis; when you understand that the second half of this sentence expands on the main idea of the first half. Making these kinds of global and local inferences is much harder for computational models, but also critical for many NLP tasks.

In the first part of my dissertation, I ask whether incorporating discourse is helpful for downstream NLP tasks. In particular, I focus on tasks that involve longer texts, which are a natural candidate to benefit from discourse, and also pose significant difficulties for models to extract and compose long-range discourse signals. I explore different forms of abstracting discourse which are somewhat effective, but are also limited by the lack of annotated data that could help them scale better. In a low-resource setting, unsupervised learning is an attractive alternative, but I find the learned structures are not representative of discourse, showing the complex inferences in discourse require stronger signals and guidance to learn. I instead attempt to manually annotate discourse in a new domain, but annotators derive conflicting yet valid discourse structures. 

This result uncovered a less studied but more critical challenge: discourse is embedded in a social context and is thus \textbf{\emph{subjective}}, but current data and models expect only a single ground truth without room for differing interpretations. I thus pioneer how to capture and model discourse subjectivity using linguistic and extra-linguistic features. 

\section{Thesis Outline}

The remainder of this thesis is structured as follows:

In Chapter~\ref{ch:background}, I review the main discourse framework I use in much of my work, as well as pragmatic theories relevant for understanding subjective interpretations in conversation.

In Chapter~\ref{ch:longertexts1}, I explore how to exploit cohesive devices as a signal of entity salience.

In Chapter~\ref{ch:longertexts2}, I experiment with embedding rhetorical discourse relations into neural networks.

In Chapter~\ref{ch:latent}, I analyze learned latent document structures which are not representative of discourse.

In Chapter~\ref{ch:annotation}, I create a dataset of discourse-segmented medical texts to explore cross-domain discourse segmentation, and discuss issues when creating discourse trees for these texts.

In Chapter~\ref{ch:subjective}, I create and analyze a dataset of subjective interpretations of conversational discourse.

Chapter~\ref{ch:conclusion} concludes this.


\section{List of Thesis Contributions}

\noindent In this thesis, I make the following contributions:

\paragraph{Coreference features for salience}
I propose a set of features based on coreference chains to extract salient information from text.

\paragraph{Discourse embeddings for rhetorical patterns}
I present a method for embedding sequences of discourse relations in a neural network to capture rhetorical patterns of salient entities.

\paragraph{Analysis of latent discourse structures}
I present several experiments and measures to prove a proposed structured attention mechanism that resolves to a dependency tree is not representative of discourse.

\paragraph{Dataset and cross-domain experiments for discourse segmentation}
I create a dataset of discourse-segmented medical texts and show cross-domain segmentation is largely successful.

\paragraph{Dataset and experiment on subjective interpretations in conversational discourse}
I create a dataset of subjective conversation acts and intents for conversational discourse along with an experiment that begins to understand the nature of this subjectivity.