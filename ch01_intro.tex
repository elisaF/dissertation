\chapter{Introduction}
\label{ch:intro}

This dissertation proposes annotation tasks and computational models to capture and predict the subjectivity of discourse.

In my completed work, I explore how existing discourse theories can help NLP end tasks. I focus on coreference as a marker of salience for an information extraction task, and on embedding rhetorical relations (in the style of Rhetorical Structure Theory (RST); \citet{Mann:1988}) for an authorship attribution task. Both experiments confirm a wealth of prior work showing that discourse is helpful on downstream NLP tasks. Motivated by the limited annotated data, I attempt to annotate discourse structures in an unsupervised fashion, but find this approach fails. I then explore manual annotation, which reveals issues with existing discourse theories, and more so with how they were operationalized. First, discourse is described only from the perspective of the writer (not the reader) which can be problematic to infer; second, the bias of the annotator is not accounted for; third, only a single label is allowed to unambiguously describe the relation type between two propositions.\footnote{PDTB 3.0 allows multiple labels, but its intent is to capture \emph{simulatenous} interpretations, not separate, alternate ones.} 

For the proposed work, I strive to address these three shortcomings. In line with the theoretical work presented in \citet{Asher:2018}, I assert discourse can be judged from the point of view of the \emph{reader}. Further, this third party reader is \emph{biased}, and their personal beliefs and preferences help determine which label they select. As a result, the discourse relation between two propositions can have \emph{multiple}, valid labels. I intend to operationalize parts of this theory by creating a new dataset and new tasks that embrace the ambiguity and subjectivity of discourse. I focus on argumentative dialogues covering polarizing topics that naturally lead to more frequent and more easily identifiable subjective interpretations. I further seek to label dialogue acts, a more intuitive task in this conversational setting. \citet{Asher:2001} show speech acts can be interpreted as rhetorical relations.

\section{Proposal Outline}
%\section{Thesis Outline}

The remainder of this thesis is structured as follows:

%In Chapter~\ref{ch:background}, ...

In Chapter~\ref{ch:completed}, I present my completed work.

In Chapter~\ref{ch:proposed}, I present my proposed work.

%Chapter~\ref{ch:conclusion} concludes this proposal.


%\section{List of Thesis Contributions}

%\noindent In this thesis, we make the following contributions:

%\paragraph{Contrib1}
%TBD

%\paragraph{Contrib2}
%TBD